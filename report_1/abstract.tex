\begin{center}
{\large {\bf  ABSTRACT }}\\
\textit{For extracting  signals, which are in small cross section compared to the large cross section of the background processes has been helped by the introduction of machine learning (ML) techniques for  classification purposes. Classification algorithms in machine learning is a type of supervised learning where the outputs are constrained only to a limited set of values or classes such as signals or backgrounds. A typical example would be the Higgs analysis. In this report, the classification of the produced resonance particles from p$\Bar{p}$ collision at Large Hadron Collider(LHC) is presented using machine learning technique trained with deep neural network(DNN). All the neural network training was done using Keras, TensorFlow, and Matplotlib in a Jupyter notebook. We will investigate all the challenges in the application of these novel analysis techniques in this report. This report concludes with a discussion of differences in the outcome of binary and multi-classification using DNN. 
}

% The Main Goal: To build a DNN to differentiate between signal and background events in an CMS data set
\end{center}  


    

